\documentclass{article}
\usepackage{tikz}
\usepackage{verbatim}
\usetikzlibrary{arrows,shapes}
\begin{document}
\pgfdeclarelayer{background}
\pgfsetlayers{background,main}
\tikzstyle{selected vertex} = [vertex, fill=red!24]
\tikzstyle{selected edge} = [draw,line width=5pt,-,red!50]\begin{center}
\begin{tabular}{| c c c |}
\hline
vertex & shortest path & length \\
\hline\hline
{$v_{2}$} & $null$ & $\infty$\\ 
{$v_{3}$} & $v_{1} \rightarrow v_{3}$ & 5\\ 
{$v_{4}$} & $null$ & $\infty$\\ 
{$v_{5}$} & $v_{1} \rightarrow v_{5}$ & 13\\ 
\hline
\end{tabular}
\end{center}
\begin{tikzpicture}[auto]
\tikzstyle{vertex}=[circle,fill=black!25,minimum size=20pt,inner sep=0pt]
\tikzstyle{edge} = [draw,thick,-]
\tikzstyle{weight} = [font=\small]
\def \radius {6cm}
\node[vertex] (1) at ({360/5 * (0 )}:\radius) {$v_{1}$};
\node[vertex] (2) at ({360/5 * (1 )}:\radius) {$v_{2}$};
\node[vertex] (3) at ({360/5 * (2 )}:\radius) {$v_{3}$};
\node[vertex] (4) at ({360/5 * (3 )}:\radius) {$v_{4}$};
\node[vertex] (5) at ({360/5 * (4 )}:\radius) {$v_{5}$};
\path[edge] (1) -- node[weight] {$5$} (3);
\path[edge] (2) -- node[weight] {$12$} (3);
\path[edge] (3) -- node[weight] {$7$} (5);
\path[edge] (3) -- node[weight] {$1$} (4);
\path[edge] (5) -- node[weight] {$13$} (1);
\path[edge] (5) -- node[weight] {$2$} (2);
\path node[selected vertex] at (1) {$v_{1}$};
\begin{pgfonlayer}{background}\path [selected edge] ( 1.center) -- (5.center);
\path [selected edge] ( 1.center) -- (3.center);
\end{pgfonlayer}\end{tikzpicture}
\begin{center}
Vertices explored from $v_{1}$. Next vertex is $v_{3}$. Edge relaxations shown in \textbf{bold}.
\end{center}
\pagebreak
\begin{center}
\begin{tabular}{| c c c |}
\hline
vertex & shortest path & length \\
\hline\hline
{$v_{2}$} & $v_{1} \rightarrow v_{3} \rightarrow v_{2}$ & 17 = min\{$\infty$, 5+12\}\\ 
{$v_{3}$} & $v_{1} \rightarrow v_{3}$ & 5\\ 
{$v_{4}$} & $v_{1} \rightarrow v_{3} \rightarrow v_{4}$ & 6 = min\{$\infty$, 5+1\}\\ 
{$v_{5}$} & $v_{1} \rightarrow v_{3} \rightarrow v_{5}$ & 12 = \textbf{min\{13, 5+7\}}\\ 
\hline
\end{tabular}
\end{center}
\begin{tikzpicture}[auto]
\tikzstyle{vertex}=[circle,fill=black!25,minimum size=20pt,inner sep=0pt]
\tikzstyle{edge} = [draw,thick,-]
\tikzstyle{weight} = [font=\small]
\def \radius {6cm}
\node[vertex] (1) at ({360/5 * (0 )}:\radius) {$v_{1}$};
\node[vertex] (2) at ({360/5 * (1 )}:\radius) {$v_{2}$};
\node[vertex] (3) at ({360/5 * (2 )}:\radius) {$v_{3}$};
\node[vertex] (4) at ({360/5 * (3 )}:\radius) {$v_{4}$};
\node[vertex] (5) at ({360/5 * (4 )}:\radius) {$v_{5}$};
\path[edge] (1) -- node[weight] {$5$} (3);
\path[edge] (2) -- node[weight] {$12$} (3);
\path[edge] (3) -- node[weight] {$7$} (5);
\path[edge] (3) -- node[weight] {$1$} (4);
\path[edge] (5) -- node[weight] {$13$} (1);
\path[edge] (5) -- node[weight] {$2$} (2);
\path node[selected vertex] at (1) {$v_{1}$};
\path node[selected vertex] at (3) {$v_{3}$};
\begin{pgfonlayer}{background}\path [selected edge] ( 3.center) -- (2.center);
\path [selected edge] ( 3.center) -- (5.center);
\path [selected edge] ( 3.center) -- (4.center);
\path [selected edge] ( 3.center) -- (1.center);
\end{pgfonlayer}\end{tikzpicture}
\begin{center}
Vertices explored from $v_{3}$. Next vertex is $v_{4}$. Edge relaxations shown in \textbf{bold}.
\end{center}
\pagebreak
\begin{center}
\begin{tabular}{| c c c |}
\hline
vertex & shortest path & length \\
\hline\hline
{$v_{2}$} & $v_{1} \rightarrow v_{3} \rightarrow v_{2}$ & 17\\ 
{$v_{3}$} & $v_{1} \rightarrow v_{3}$ & 5 = min\{5, 6+1\}\\ 
{$v_{4}$} & $v_{1} \rightarrow v_{3} \rightarrow v_{4}$ & 6\\ 
{$v_{5}$} & $v_{1} \rightarrow v_{3} \rightarrow v_{5}$ & 12\\ 
\hline
\end{tabular}
\end{center}
\begin{tikzpicture}[auto]
\tikzstyle{vertex}=[circle,fill=black!25,minimum size=20pt,inner sep=0pt]
\tikzstyle{edge} = [draw,thick,-]
\tikzstyle{weight} = [font=\small]
\def \radius {6cm}
\node[vertex] (1) at ({360/5 * (0 )}:\radius) {$v_{1}$};
\node[vertex] (2) at ({360/5 * (1 )}:\radius) {$v_{2}$};
\node[vertex] (3) at ({360/5 * (2 )}:\radius) {$v_{3}$};
\node[vertex] (4) at ({360/5 * (3 )}:\radius) {$v_{4}$};
\node[vertex] (5) at ({360/5 * (4 )}:\radius) {$v_{5}$};
\path[edge] (1) -- node[weight] {$5$} (3);
\path[edge] (2) -- node[weight] {$12$} (3);
\path[edge] (3) -- node[weight] {$7$} (5);
\path[edge] (3) -- node[weight] {$1$} (4);
\path[edge] (5) -- node[weight] {$13$} (1);
\path[edge] (5) -- node[weight] {$2$} (2);
\path node[selected vertex] at (1) {$v_{1}$};
\path node[selected vertex] at (3) {$v_{3}$};
\path node[selected vertex] at (4) {$v_{4}$};
\begin{pgfonlayer}{background}\path [selected edge] ( 4.center) -- (3.center);
\end{pgfonlayer}\end{tikzpicture}
\begin{center}
Vertices explored from $v_{4}$. Next vertex is $v_{5}$. Edge relaxations shown in \textbf{bold}.
\end{center}
\pagebreak
\begin{center}
\begin{tabular}{| c c c |}
\hline
vertex & shortest path & length \\
\hline\hline
{$v_{2}$} & $v_{1} \rightarrow v_{3} \rightarrow v_{5} \rightarrow v_{2}$ & 14 = \textbf{min\{17, 12+2\}}\\ 
{$v_{3}$} & $v_{1} \rightarrow v_{3}$ & 5 = min\{5, 12+7\}\\ 
{$v_{4}$} & $v_{1} \rightarrow v_{3} \rightarrow v_{4}$ & 6\\ 
{$v_{5}$} & $v_{1} \rightarrow v_{3} \rightarrow v_{5}$ & 12\\ 
\hline
\end{tabular}
\end{center}
\begin{tikzpicture}[auto]
\tikzstyle{vertex}=[circle,fill=black!25,minimum size=20pt,inner sep=0pt]
\tikzstyle{edge} = [draw,thick,-]
\tikzstyle{weight} = [font=\small]
\def \radius {6cm}
\node[vertex] (1) at ({360/5 * (0 )}:\radius) {$v_{1}$};
\node[vertex] (2) at ({360/5 * (1 )}:\radius) {$v_{2}$};
\node[vertex] (3) at ({360/5 * (2 )}:\radius) {$v_{3}$};
\node[vertex] (4) at ({360/5 * (3 )}:\radius) {$v_{4}$};
\node[vertex] (5) at ({360/5 * (4 )}:\radius) {$v_{5}$};
\path[edge] (1) -- node[weight] {$5$} (3);
\path[edge] (2) -- node[weight] {$12$} (3);
\path[edge] (3) -- node[weight] {$7$} (5);
\path[edge] (3) -- node[weight] {$1$} (4);
\path[edge] (5) -- node[weight] {$13$} (1);
\path[edge] (5) -- node[weight] {$2$} (2);
\path node[selected vertex] at (1) {$v_{1}$};
\path node[selected vertex] at (3) {$v_{3}$};
\path node[selected vertex] at (4) {$v_{4}$};
\path node[selected vertex] at (5) {$v_{5}$};
\begin{pgfonlayer}{background}\path [selected edge] ( 5.center) -- (1.center);
\path [selected edge] ( 5.center) -- (2.center);
\path [selected edge] ( 5.center) -- (3.center);
\end{pgfonlayer}\end{tikzpicture}
\begin{center}
Vertices explored from $v_{5}$. Next vertex is $v_{2}$. Edge relaxations shown in \textbf{bold}.
\end{center}
\pagebreak
\begin{center}
\begin{tabular}{| c c c |}
\hline
vertex & shortest path & length \\
\hline\hline
{$v_{2}$} & $v_{1} \rightarrow v_{3} \rightarrow v_{5} \rightarrow v_{2}$ & 14\\ 
{$v_{3}$} & $v_{1} \rightarrow v_{3}$ & 5 = min\{5, 14+12\}\\ 
{$v_{4}$} & $v_{1} \rightarrow v_{3} \rightarrow v_{4}$ & 6\\ 
{$v_{5}$} & $v_{1} \rightarrow v_{3} \rightarrow v_{5}$ & 12 = min\{12, 14+2\}\\ 
\hline
\end{tabular}
\end{center}
\begin{tikzpicture}[auto]
\tikzstyle{vertex}=[circle,fill=black!25,minimum size=20pt,inner sep=0pt]
\tikzstyle{edge} = [draw,thick,-]
\tikzstyle{weight} = [font=\small]
\def \radius {6cm}
\node[vertex] (1) at ({360/5 * (0 )}:\radius) {$v_{1}$};
\node[vertex] (2) at ({360/5 * (1 )}:\radius) {$v_{2}$};
\node[vertex] (3) at ({360/5 * (2 )}:\radius) {$v_{3}$};
\node[vertex] (4) at ({360/5 * (3 )}:\radius) {$v_{4}$};
\node[vertex] (5) at ({360/5 * (4 )}:\radius) {$v_{5}$};
\path[edge] (1) -- node[weight] {$5$} (3);
\path[edge] (2) -- node[weight] {$12$} (3);
\path[edge] (3) -- node[weight] {$7$} (5);
\path[edge] (3) -- node[weight] {$1$} (4);
\path[edge] (5) -- node[weight] {$13$} (1);
\path[edge] (5) -- node[weight] {$2$} (2);
\path node[selected vertex] at (1) {$v_{1}$};
\path node[selected vertex] at (3) {$v_{3}$};
\path node[selected vertex] at (4) {$v_{4}$};
\path node[selected vertex] at (5) {$v_{5}$};
\path node[selected vertex] at (2) {$v_{2}$};
\begin{pgfonlayer}{background}\path [selected edge] ( 2.center) -- (3.center);
\path [selected edge] ( 2.center) -- (5.center);
\end{pgfonlayer}\end{tikzpicture}
\begin{center}
Vertices explored from $v_{2}$. Next vertex is $v_{2}$. Edge relaxations shown in \textbf{bold}.
\end{center}
\pagebreak
\end{document}
